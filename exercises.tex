% Notes and exercises from Introduction to Real Analysis by DePree and Swartz
% By John Peloquin
\documentclass[letterpaper,12pt]{article}
\usepackage{amsmath,amssymb,amsthm,enumitem,fourier}

\newcommand{\R}{\mathbf{R}}
\newcommand{\Rp}{\R_+}
\newcommand{\Rps}{\Rp^*}

\newcommand{\after}{\circ}

\newcommand{\inv}[1]{#1^{-1}}

% Theorems
\theoremstyle{plain}
\newtheorem*{prop}{}

\theoremstyle{definition}
\newtheorem*{exer}{Exercise}

\theoremstyle{remark}
\newtheorem*{rmk}{Remark}

% Meta
\title{Notes and exercises from\\\textit{Foundations of Modern Analysis}}
\author{John Peloquin}
\date{}

\begin{document}
\maketitle

\section*{Introduction}
This document contains notes and exercises from~\cite{dieudonne}.

\section*{Chapter~IV}
\subsection*{Section~3}
We provide an alternative proof of~(4.3.7):

\begin{prop}[4.3.7]
Any continuous mapping~\(g\) of~~\(\Rps\) into itself such that \(g(xy)=g(x)g(y)\) has the form \(x\mapsto x^a\), with \(a\in\R\).
\end{prop}
\begin{proof}
If \(b>1\), then \(f=\log_b\after g:\Rps\to\R\) is continuous and
\[f(xy)=\log_b g(xy)=\log_b[g(x)g(y)]=\log_b g(x)+\log_b g(y)=f(x)+f(y)\]
If \(g\ne1\), there is \(x>0\) with \(g(x)>1\) (if \(g(x)<1\), then \(g(\inv{x})=\inv{g(x)}>1\)), so \(y_n=g(x^n)=g(x)^n\to\infty\) as \(n\to\infty\) and \(y_n\to 0\) as \(n\to-\infty\). It follows from (3.19.1) and~(3.19.7) that \(g\)~is surjective, so there is \(c>0\) with \(c\ne 1\) such that \(g(c)=b\) and hence \(f(c)=1\). By~(4.3.2), \(f=\log_c\), so
\[g(x)=b^{\log_c x}=b^{a\log_b x}=x^a\]
where \(a=\log_c b\), by (4.3.3) and~(4.3.4).
\end{proof}
\begin{rmk}
This proof is longer than Dieudonn\'e's, but avoids direct use of~(4.1.3), which is already used in the proof of~(4.3.2).
\end{rmk}

% References
\begin{thebibliography}{0}
\bibitem{dieudonne} Dieudonn\'e, J. \textit{Foundations of Modern Analysis.} Academic Press, 1960.
\end{thebibliography}
\end{document}
