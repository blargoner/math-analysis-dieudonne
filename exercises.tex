% Notes and exercises from Introduction to Real Analysis by DePree and Swartz
% By John Peloquin
\documentclass[letterpaper,12pt]{article}
\usepackage{amsmath,amssymb,amsthm,enumitem,fourier}

\newcommand{\N}{\mathbf{N}}
\newcommand{\Nex}{\overline{\N}}
\newcommand{\R}{\mathbf{R}}
\newcommand{\Rp}{\R_+}
\newcommand{\Rps}{\Rp^*}
\newcommand{\Rex}{\overline{\R}}
\newcommand{\C}{\mathbf{C}}
\renewcommand{\Re}{\mathcal{R}}
\newcommand{\B}{\mathcal{B}}
\newcommand{\BC}{\mathcal{C}^{\infty}}
\renewcommand{\L}{\mathcal{L}}

\newcommand{\union}{\cup}
\newcommand{\sect}{\cap}
\newcommand{\after}{\circ}
\newcommand{\downto}{\downarrow}
\newcommand{\iso}{\cong}

\newcommand{\conj}[1]{\overline{#1}}
\newcommand{\closure}[1]{\overline{#1}}
\newcommand{\inv}[1]{#1^{-1}}
\newcommand{\abs}[1]{|{#1}|}
\newcommand{\norm}[1]{\lVert{#1}\rVert}
\newcommand{\innerprod}[2]{({#1}|{#2})}
\newcommand{\cbprod}[2]{[{#1}\cdot{#2}]}
\renewcommand{\d}[1]{\,d\!{#1}}
\newcommand{\dx}{\d{x}}

% Theorems
\theoremstyle{plain}
\newtheorem*{prop}{}

\theoremstyle{definition}
\newtheorem*{exer}{Exercise}

\theoremstyle{remark}
\newtheorem*{rmk}{Remark}

% Meta
\title{Notes and exercises from\\\textit{Foundations of Modern Analysis}}
\author{John Peloquin}
\date{}

\begin{document}
\maketitle

\section*{Introduction}
This document contains notes and exercises from~\cite{dieudonne}.

\section*{Chapter~IV}
\subsection*{Section~3}
\begin{rmk}
In the proof of~(4.3.1), intuitively we know that \(f=\log_a\). Recall by the density of the rationals in the reals (2.2.16) that \emph{a real number is the supremum of the set of rationals less than it}. Therefore \(\log_a x\)~is just the supremum of the set of rationals~\(m/n\) with \(m/n\le\log_a x\), or equivalently \(a^{m/n}\le x\), or equivalently \(a^m\le x^n\); that is, \(\log_a x=\sup A_x\). Since neither \(\log_a x\) nor~\(a^{m/n}\) are defined yet, but \(a^m\)~and~\(x^n\) are, it is natural to show that \(f(x)=\sup A_x\) for uniqueness.

For existence, we know the set~\(A_x\) is well defined. Fix \(k\ge 1\) and choose~\(p,q\) by~(4.3.1.1) with \(a^p\le x^k\le a^q\). Then \(p/k\in A_x\), so \(A_x\)~is nonempty. If \(m/n\in A_x\) (\(n\ge 1\)), then \(a^m\le x^n\), so \(a^{mk}\le x^{nk}\le a^{qn}\), so \(mk\le qn\), and \(m/n\le q/k\); that is, \(A_x\)~is bounded above by~\(q/k\). Therefore \(f(x)=\sup A_x\)~is defined, and \(p/k\le f(x)\le q/k\). Taking \(q=p+1\) yields \(p/k\le f(x)\le(p+1)/k\), which is used in the proof that \(f(xy)=f(x)+f(y)\). Taking \(x=a\) and \(k=p=q=1\) yields \(f(a)=1\).


\end{rmk}

\noindent We provide an alternative proof of~(4.3.7):
\begin{prop}[4.3.7]
Any continuous mapping~\(g\) of~~\(\Rps\) into itself such that \(g(xy)=g(x)g(y)\) has the form \(x\mapsto x^a\), with \(a\in\R\).
\end{prop}
\begin{proof}
If \(b>1\), then \(f=\log_b\after g:\Rps\to\R\) is continuous and
\[f(xy)=\log_b g(xy)=\log_b[g(x)g(y)]=\log_b g(x)+\log_b g(y)=f(x)+f(y)\]
If \(g\ne1\), there is \(x>0\) with \(g(x)>1\) (if \(g(x)<1\), then \(g(\inv{x})=\inv{g(x)}>1\)), so \(y_n=g(x^n)=g(x)^n\to\infty\) as \(n\to\infty\) and \(y_n\to 0\) as \(n\to-\infty\). It follows from (3.19.1) and~(3.19.7) that \(g\)~is surjective, so there is \(c>0\) with \(c\ne 1\) such that \(g(c)=b\) and hence \(f(c)=1\). By~(4.3.2), \(f=\log_c\), so
\[g(x)=b^{\log_c x}=b^{a\log_b x}=x^a\]
where \(a=\log_c b\), by (4.3.3) and~(4.3.4).
\end{proof}
\begin{rmk}
This proof is longer than Dieudonn\'e's, but avoids direct use of~(4.1.3), which is already used in the proof of~(4.3.2).
\end{rmk}

\subsection*{Section~5}
\begin{rmk}
In the proof of the Tietze-Urysohn extension theorem~(4.5.1), the idea behind the formula defining~\(g(x)\) for \(x\in E-A\) is roughly: \emph{for \(x\)~very near the boundary (frontier) of~\(A\), take the limit of~\(f(y)\) as \(y\in A\) approaches points near~\(x\)}:
\[g(x)=\inf_{y\in A}\left[f(y)\frac{d(x,y)}{d(x,A)}\right]\qquad(x\in E-A)\]
Write \(\rho_x(y)=d(x,y)/d(x,A)\). Since \(1\le f(y)\le 2\) by the reduction, \(\rho_x(y)\ge 1\), \(\rho_x(y)>2\) for \(y\)~sufficiently far from~\(x\), and \(\rho_x(y)\to 1\) as \(y\to x\), it follows that \(\inf_{y\in A}f(y)\rho_x(y)\) reflects values of~\(f(y)\) at \(y\)~nearest to~\(x\). This ensures continuity of~\(g\) on the boundary of~\(A\). Since \(g\)~is also continuous in the interior of~\(A\) (by continuity of~\(f\)), and in the exterior of~\(A\) (by continuity of the metric~\(d\)), \(g\)~is continuous on~\(E\).
\end{rmk}

\section*{Chapter~V}
\subsection*{Section~2}
\begin{rmk}
(5.2.4)~is an associativity property for convergent series in a normed space. For \emph{absolutely} convergent series in a \emph{Banach} space, (5.3.6)~is a stronger associativity property. For example, if \(\sum_{n=0}^{\infty}x_n\) is absolutely convergent in~\(\C\), it follows from the remark below~(5.3.6), but not from~(5.2.4), that \(\sum_{n=0}^{\infty}x_n=\sum_{n=0}^{\infty}x_{2n}+\sum_{n=0}^{\infty}x_{2n+1}\).
\end{rmk}

\subsection*{Section~3}
\begin{rmk}
In the proof of~(5.3.2), to see how the inequality \(\norm{\sum_{n=0}^{\infty}x_n}\le\sum_{n=0}^{\infty}\norm{x_n}\) follows from the principle of extension of inequalities~(3.15.4), recall from the definition of sequential limit in~(3.13) that the functions given by
\[s(n)=\sum_{k=0}^n x_k\quad s(\infty)=\sum_{k=0}^{\infty}x_k\qquad\text{and}\qquad t(n)=\sum_{k=0}^n\norm{x_k}\quad t(\infty)=\sum_{k=0}^{\infty}\norm{x_k}\]
are \emph{continuous} on \(\N\union\{\infty\}=\Nex\subseteq\Rex\). By continuity of the norm, the composite function \(n\mapsto\norm{s(n)}\) is also continuous on~\(\Nex\). Now \(\norm{s(n)}\le t(n)\) for all \(n\in\N\) and \(\N\)~is dense in~\(\Nex\), hence \(\norm{s(\infty)}\le t(\infty)\) by~(3.15.4), as desired. Other applications of the principle of extension are similar.
\end{rmk}

\begin{rmk}
In~(5.3.4), \(\sum_{\alpha\in A}\norm{x_{\alpha}}=\sup\left\{\,\sum_{\alpha\in J}\norm{x_{\alpha}}\mid J\subseteq A\text{ finite}\,\right\}\) by~(5.3.1). This is implicit in the proof of~(5.3.5).
\end{rmk}

\begin{rmk}
In~(5.3.4), \(2\epsilon\)~can be replaced by~\(\epsilon\), since \(H\)~can be chosen for~\(\epsilon/2\).
\end{rmk}

\begin{rmk}
In the proof of~(5.3.4), to construct~\(H\), first fix a bijection \(\varphi:\N\to A\). Then \(\sum_{n=0}^{\infty}\norm{x_{\varphi(n)}}\) converges, so there exists~\(M\) such that for all \(n\ge m\ge M\), \(\sum_{k=m}^n\norm{x_{\varphi(k)}}\le\epsilon\) (5.2.1). Let \(H=\varphi[0,M)\). Then if \(K\subseteq A\) is finite with \(H\sect K=\emptyset\), \(\inv{\varphi}[K]\subseteq[M,\infty)\), so if \(N=\max\inv{\varphi}[K]\), then \(\sum_{\alpha\in K}\norm{x_{\alpha}}\le\sum_{k=M}^N\norm{x_k}\le\epsilon\). Also \(\norm{\sum_{\alpha\in A}x_{\alpha}-\sum_{\alpha\in H}x_{\alpha}}=\norm{\sum_{n=M}^{\infty}x_{\varphi(n)}}\le\sum_{n=M}^{\infty}\norm{x_{\varphi(n)}}\le\epsilon\) by (5.3.1)~and~(5.3.2). Finally, if \(L\)~is finite with \(H\subseteq L\subseteq A\), then \(L-H\)~is finite with \((L-H)\sect H=\emptyset\), so \(\norm{\sum_{\alpha\in H}x_{\alpha}-\sum_{\alpha\in L}x_{\alpha}}=\norm{\sum_{\alpha\in L-H}x_{\alpha}}\le\sum_{\alpha\in L-H}\norm{x_{\alpha}}\le\epsilon\), and it follows that \(\norm{\sum_{\alpha\in A}x_{\alpha}-\sum_{\alpha\in L}x_{\alpha}}\le 2\epsilon\).
\end{rmk}

\begin{rmk}
In the proof of~(5.3.6), the idea is to use finite approximations of finitely many of the~\(z_n\) to obtain a finite approximation of~\(\sum_{\alpha\in A}x_\alpha\).
\end{rmk}

\subsection*{Section~5}
\begin{rmk}
In a Banach space, \(\lambda\sum_{n=0}^{\infty}x_n=\sum_{n=0}^{\infty}\lambda x_n\) for any scalar~\(\lambda\), when either series converges, by (5.1.5)~and~(5.5.2).
\end{rmk}

\begin{rmk}
Let \(L_1(I)\)~be the space of Lebesgue integrable real-valued functions\footnote{Technically this space consists of equivalence classes of functions.} on the interval \(I\subseteq\R\) with norm \(\norm{f}_1=\int_I\abs{f}\). Then \(L_1(I)\)~is a Banach space (Riesz-Fischer). The integral operator \(f\mapsto\int_If\) is linear and continuous since \(\abs{\int_If}\le\int_I\abs{f}\) (5.5.1). Therefore if \(f=\sum_{n=0}^{\infty}f_n\) under convergence in the norm, then \(\int_I f=\sum_{n=0}^{\infty}\int_I f_n\) (5.5.2).
\end{rmk}

\begin{rmk}
The multiplication operation \((x,y)\mapsto xy\) from~\(\C^2\) to~\(\C\) is bilinear and continuous by~(4.4.1), hence if \(\sum_{n=0}^{\infty}x_n\) and \(\sum_{n=0}^{\infty}y_n\) are absolutely convergent series of complex numbers, then \((\sum_{n=0}^{\infty}x_n)(\sum_{n=0}^{\infty}y_n)=\sum_{m,n}x_my_n\), where the latter sum may be taken \emph{in any order}, by~(5.5.3).
\end{rmk}

\begin{rmk}
In the proof of~(5.5.4), to see how the linearity of~\(\closure{f}\) follows from the principle of extension of identities~(3.15.2), observe that the mappings
\[(x,y)\mapsto x+y\mapsto\closure{f}(x+y)\qquad\text{and}\qquad(x,y)\mapsto(\closure{f}(x),\closure{f}(y))\mapsto\closure{f}(x)+\closure{f}(y)\]
are continuous by (3.11.5), (3.20.15), and~(5.1.5), and agree on~\(G^2\), which is dense in~\(E^2\) by~(3.20.3), and hence agree on~\(E^2\) by~(3.15.2). Other applications of the principle of extension are similar.
\end{rmk}

\subsection*{Section~8}
\begin{rmk}
Intuitively, a \emph{closed} hyperplane separates a vector space into the two half-spaces on either side of the hyperplane, but a \emph{dense} hyperplane does not have this property.
\end{rmk}

\section*{Chapter~VI}
\subsection*{Section~3}
\begin{rmk}
(6.3.1.1)~is the \emph{parallelogram law}, which states that the sum of the squares of the lengths of the diagonals of a parallelogram is equal to the sum of the squares of the lengths of its sides.
\end{rmk}

\begin{rmk}
In the proof of~(6.3.1),
\begin{align*}
\norm{x-(y+\lambda z)}^2&=\innerprod{x-y-\lambda z}{x-y-\lambda z}\\
	&=\norm{x-y}^2-\lambda[\innerprod{x-y}{z}+\innerprod{z}{x-y}]+\lambda^2\norm{z}^2\\
	&=\alpha^2-2\lambda\Re\innerprod{x-y}{z}+\lambda^2\norm{z}^2>\alpha^2
\end{align*}
so
\[-2\lambda\Re\innerprod{x-y}{z}+\lambda^2\norm{z}^2>0\tag{1}\]
Substituting~\(-\lambda\) for~\(\lambda\) in~(1) yields
\[2\lambda\Re\innerprod{x-y}{z}+\lambda^2\norm{z}^2>0\tag{2}\]
Now (1)~and~(2) imply
\[\abs{\Re\innerprod{x-y}{z}}<\lambda\frac{\norm{z}^2}{2}\qquad(\lambda>0)\]
Letting \(\lambda\downto 0\) shows \(\Re\innerprod{x-y}{z}=0\).
\end{rmk}

\subsection*{Section~5}
\begin{rmk}
If \(E\)~is the Hilbert sum of a normal system~\((a_n)\) and \(x=(x_n)\in E\), then \(x_n=\lambda_n a_n\) with
\[\innerprod{x}{j_n(a_n)}=\sum_{m=1}^{\infty}\innerprod{\lambda_ma_m}{\delta_{mn}a_m}=\lambda_n\norm{a_n}^2=\lambda_n\]
It follows that \(x=\sum_{n=1}^{\infty}\innerprod{x}{j_n(a_n)}j_n(a_n)\) and this representation in~\((j_n(a_n))\) is unique. If \(y\in E\),
\[\innerprod{x}{y}=\sum_{n=1}^{\infty}\innerprod{x}{j_n(a_n)}\conj{\innerprod{y}{j_n(a_n)}}\]
In particular \(\norm{x}^2=\sum_{n=1}^{\infty}\abs{\innerprod{x}{j_n(a_n)}}^2\).
\end{rmk}

\begin{rmk}
In the proof of~(6.5.2), to see how the result follows from~(6.4.2), let \(\varphi\)~be the isomorphism of~\(F\) onto the Hilbert sum~\(E\) of the~\(a_n\) with \(\varphi(a_n)=j_n(a_n)\). If \(x\in F\), by the previous remark and properties of~\(\varphi\),
\[\varphi(x)=\sum_{n=1}^{\infty}\innerprod{\varphi(x)}{\varphi(a_n)}\varphi(a_n)=\varphi\left[\sum_{n=1}^{\infty}\innerprod{x}{a_n}a_n\right]\]
so \(x=\sum_{n=1}^{\infty}\innerprod{x}{a_n}a_n\). If \(y\in F\), then
\[\innerprod{x}{y}=\innerprod{\varphi(x)}{\varphi(y)}=\sum_{n=1}^{\infty}\innerprod{\varphi(x)}{\varphi(a_n)}\conj{\innerprod{\varphi(y)}{\varphi(a_n)}}=\sum_{n=1}^{\infty}\innerprod{x}{a_n}\conj{\innerprod{y}{a_n}}\]
In particular \(\norm{x}^2=\sum_{n=1}^{\infty}\abs{\innerprod{x}{a_n}}^2\). If \((\lambda_n)\)~is a sequence of scalars with \(\sum_{n=1}^{\infty}\abs{\lambda_n}^2\) convergent, then \(z=(\lambda_na_n)\in E\) and \(x=\inv{\varphi}(z)\) is unique such that \(\innerprod{x}{a_n}=\innerprod{z}{\varphi(a_n)}=\lambda_n\).
\end{rmk}

\section*{Chapter~VII}
\subsection*{Section~1}
\begin{rmk}
In~(7.1.2.1), the mappings~\(f_i\) are the \emph{scalar component functions} of~\(f\), and the mappings \(t\mapsto f_i(t)a_i\) are the \emph{vector component functions} of~\(f\).

(7.1.2)~shows that if \(F\)~is finite-dimensional, then \(\B_F(A)\)~decomposes into subspaces of vector component functions, each of which is isomorphic to a space of scalar component functions. By~(5.9.1), this result can be expressed in the equivalent form
\[\B_{K^n}(A)\iso\B_K(A)^n\]
where \(K=\R\) or \(K=\C\).
\end{rmk}

\subsection*{Section~2}
\begin{rmk}
If \(E\)~is a metric space and \(K=\R\) or \(K=\C\), then \(\BC_{K^n}(E)\iso\BC_K(E)^n\).
\end{rmk}

\begin{rmk}
If \(E\)~is a metric space and \(F\)~is a Banach space, then \(\BC_F(E)\)~is a Banach space, by (7.2.1), (7.1.3), and~(3.14.5).
\end{rmk}

\begin{rmk}
The proof of Dini's theorem~(7.2.2) is like the proof of~(7.2.1) ``turned on its side.''
\end{rmk}

\subsection*{Section~5}
\begin{rmk}
(7.5.6) is like Dini's theorem~(7.2.2), but with equicontinuity replacing monotonicity and continuity.
\end{rmk}

\subsection*{Section~6}
\begin{rmk}
If \(f:I\to F\) is regulated and \(g:F\to G\) is continuous, then \(g\after f:I\to G\) is regulated, by~(3.13.6). In particular, \(\xi\mapsto\norm{f(\xi)}\)~is regulated.
\end{rmk}

\begin{rmk}
If \(f:I\to\R\) is monotone and \(g:J\to F\) is regulated with \(f(I)\subseteq J\), then \(g\after f:I\to F\) is regulated, by (4.2.1) and~(3.13.6).
\end{rmk}

\begin{rmk}
If \(f_i:I\to F_i\) is regulated for all \(1\le i\le n\), then \(f=(f_i):I\to\prod F_i\) is regulated, by~(3.20.5). It follows that \(\sum f_i\)~is regulated, by (5.1.5)~and the first remark above.
\end{rmk}

\begin{rmk}
The uniform limit of a sequence of regulated functions on a compact interval is regulated, by~(7.6.1).
\end{rmk}

\section*{Chapter~VIII}
\subsection*{Section~6}
\begin{rmk}
If a function~\(f\) is differentiable at a point~\(x_0\), then change in~\(f\) \emph{at~\(x_0\)} is well approximated by~\(f'(x_0)\). If \(f\)~is \emph{continuously} differentiable at~\(x_0\), then change in~\(f\) \emph{between any two points sufficiently close to~\(x_0\)} is well approximated by~\(f'(x_0)\), by (8.6.2).\footnote{See also Problem~3.}
\end{rmk}

\subsection*{Section~7}
\begin{rmk}
By~(7.6.1), a regulated function \(f:I\to\R\), where \(I\subseteq\R\) is a compact interval, is uniformly approximated by real-valued step functions, and hence intuitively has a well defined ``area under its curve'', namely the limit of the areas under its approximating step functions.

On the other hand, in the proof of~(8.7.2), we see that a primitive of a step function gives the area under its curve, and since the primitive of~\(f\) is just the limit of the primitives of its approximating step functions (8.6.4), the primitive of~\(f\) gives the area under~\(f\). This motivates the definition of the integral in terms of primitives (the Cauchy integral).
\end{rmk}

\begin{rmk}
The Cauchy integral makes one half of the fundamental theorem of calculus true by definition; the other half is~(8.7.3).
\end{rmk}

\begin{rmk}
In the proof of~(8.7.3), to see how the inequality follows from~(8.5.2), let \(h(x)=g(x)-f(\xi)x\). Since \(g\)~is a primitive of~\(f\), there is a denumerable set \(D\subseteq I\) such that for all \(x\in I-D\), \(h'(x)=f(x)-f(\xi)\). It follows from~(8.5.2) that for any \(0\le\zeta\le\lambda\),
\[\norm{h(\xi+\zeta)-h(\xi)}\le\zeta\sup_{0\le\eta\le\lambda}\norm{f(\xi+\eta)-f(\xi)}\]
as desired. This is like the proof of~(8.6.2) from~(8.5.4).
\end{rmk}

\begin{rmk}
In the proof of~(8.7.4), to see that \(\xi\mapsto f(\varphi(\xi))\varphi'(\xi)\) is regulated, first note that if \(f\)~is continuous, then \(f\after\varphi\)~is continuous by~(3.11.5), and hence regulated, by~(7.6.2); on the other hand if \(\varphi\)~is monotone, then \(f\after\varphi\)~is regulated, by a remark from~(7.6) above. Since \(\varphi'\)~is also regulated,
\[\xi\mapsto(f(\varphi(\xi)),\varphi'(\xi))\mapsto f(\varphi(\xi))\varphi'(\xi)\]
is regulated, by~(5.1.5) and remarks from~(7.6) above.
\end{rmk}

\begin{rmk}
In the proof of~(8.7.5), note for example
\[\xi\mapsto(f(\xi),g'(\xi))\mapsto\cbprod{f(\xi)}{g'(\xi)}\]
is regulated, by (7.6.2)~and remarks from~(7.6) above. Also note use of additivity of the integral in the proof, which follows from~(8.2.2).
\end{rmk}

\begin{rmk}
In the proof of~(8.7.6), \(u\after f\)~is regulated, by a remark from~(7.6) above.
\end{rmk}

\begin{rmk}
In the proof of the mean value theorem for integrals~(8.7.7), \(\xi\mapsto\norm{f(\xi)}\) is regulated, by a remark from~(7.6) above. If \(g\)~is a primitive of~\(f\) and \(h\)~is a primitive of \(\xi\mapsto\norm{f(\xi)}\) (8.7.2), then for all~\(\xi\) not in some denumerable set~\(D\), \(\norm{g'(\xi)}\le h'(\xi)\), hence
\[\norm{g(\beta)-g(\alpha)}\le h(\beta)-h(\alpha)\le(\beta-\alpha)\sup_{\xi\in I-D} h'(\xi)\]
by (8.5.1) and~(8.5.2), as desired.
\end{rmk}

\subsection*{Section~8}
\begin{rmk}
The function \(1/x\)~is regulated on~\([1,a]\) by (4.1.4) and~(7.6.2), and
\[\int_1^a1/x\,\dx=\log a-\log 1=\log a\]
since \(\log x\)~is a primitive of~\(1/x\). This integral can serve as a definition of~\(\log a\).
\end{rmk}

\subsection*{Section~9}
\begin{rmk}
In the remarks above~(8.9.1), the partial mappings \(x_1\mapsto f(x_1,a_2)\) and \(x_2\mapsto f(a_1,x_2)\) are continuous on open subsets of \(E_1\) and~\(E_2\), respectively, by (3.20.12) and~(3.20.14), so it makes sense to talk about their differentiability.
\end{rmk}

\begin{rmk}
In the proof of~(8.9.1), note (8.9.1.1)~holds as long as \(f\)~is \emph{differentiable} at~\((x_1,x_2)\); it implies
\[Df(x_1,x_2)=D_1f(x_1,x_2)\after p_1+D_2f(x_1,x_2)\after p_2\]
which implies\footnote{The equation in the book is incorrect.}
\[Df=P_1\after D_1f+P_2\after D_2f\]
where \(P_i(\varphi)=\varphi\after p_i\). Now \(P_i\in\L(\L(E_i,F),\L(E_1\times E_2,F))\) by (5.7.5) and~(5.5.1), so \(Df\)~is continuous if \(D_1f\) and~\(D_2f\) are.
\end{rmk}

\begin{rmk}
In~(8.9.2), let \(g=(g_i)\) and \(h=f\after g\), so \(g\)~is differentiable by~(8.1.5) with \(Dg(x)=(Dg_i(x))\) and \(h\)~is differentiable by the chain rule~(8.2.1) with \(Dh(x)=Df(g(x))\after Dg(x)\). It follows from~(8.9.1) that\footnote{The equation in the book is incorrect.}
\[Dh(x)=\sum_{i=1}^nD_if(g(x))\after Dg_i(x)\]
This holds as long as \(f\)~and the~\(g_i\) are \emph{differentiable} (by the previous remark), but if in addition \(Df\)~and the~\(Dg_i\) are continuous, then \(Dh\)~is continuous (by reasoning as in the previous remark).
\end{rmk}

\subsection*{Section~10}
\begin{rmk}
In~(8.10.1), let \(K\)~be \(\R\) or~\(\C\), \(\varphi=(\varphi_i):K^n\to K^m\), \(\psi=(\psi_i):K^m\to K^p\), and \(\theta=(\theta_i)=\psi\after\varphi:K^n\to K^p\). Then by the chain rule~(8.2.1), \(\theta\)~is differentiable and \(D\theta(x)=D\psi(\varphi(x))\after D\varphi(x)\). It follows from the remarks above~(8.10.1) that\footnote{The equation in the book is incorrect.}
\[(D_k\theta_i(x))=(D_j\psi_i(\varphi(x)))(D_k\varphi_j(x))\]
and similarly for the determinants when \(m=n=p\). This holds as long as the functions \(\varphi_i\) and~\(\psi_i\) are \emph{differentiable}, by a remark from~(8.9) above.
\end{rmk}

% References
\begin{thebibliography}{0}
\bibitem{dieudonne} Dieudonn\'e, J. \textit{Foundations of Modern Analysis.} Academic Press, 1960.
\end{thebibliography}
\end{document}
